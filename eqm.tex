% Created 2021-02-13 Sat 15:00
\documentclass[a4paper,12pt]{article}
\usepackage[utf8]{inputenc}
\usepackage[T1]{fontenc}
\usepackage{graphicx}
\usepackage{grffile}
\usepackage{longtable}
\usepackage{wrapfig}
\usepackage{rotating}
\usepackage[normalem]{ulem}
\usepackage{amsmath}
\usepackage{textcomp}
\usepackage{amssymb}
\usepackage{capt-of}
\usepackage{hyperref}
\usepackage{fullpage}
%
% American mathematical society libs
%

% imports
\usepackage{amsmath,amssymb,latexsym,amsfonts,amsthm, mathtools}

% definitions
\newtheorem{theorem}{Theorem}[section]
\newtheorem{lemma}{Lemma}[section]
\newtheorem{proposition}{Proposition}[theorem]
\newtheorem{corollary}{Corollary}[theorem]

\theoremstyle{definition}
\newtheorem{definition}{Definition}[section]
\newtheorem{example}{Example}[section]
\newtheorem{remark}{Remark}[section]

%
% Commands and aliases
%

% Complex numbers set
\newcommand{\C}{\mathbb{C}}

% Real numbers set
\newcommand{\R}{\mathbb{R}}
\newcommand{\Rn}{\mathbb{R}^n}


% Rational numbers set
\newcommand{\Q}{\mathbb{Q}}

% Integer numbers set
\newcommand{\Z}{\mathbb{Z}}

% Natural numbers set
\newcommand{\N}{\mathbb{N}}

% Prime numbers set
\renewcommand{\P}{\mathbb{P}}

% Ring of integers set
\renewcommand{\O}{\mathcal{O}}
\newcommand{\Ok}{\mathcal{O}_K}

% Ring of integers set
\newcommand{\Id}{\mathfrak{I}}

% Canonical basis
\newcommand{\Cb}{\mathcal{C}}


% Definition
\newcommand{\defsym}{\vcentcolon =}

% Id Est i.e.
\newcommand{\ie}{\textit{i.e.}}

% Maximum real subfield
\newcommand{\maxrs}{\Q(\zeta_{p} + \zeta_{p} ^{-1})}

% K_\R inner space
\newcommand{\krspace}{(K_\R,\langle{\cdot,\cdot}\rangle_\tau)}
\usepackage[utf8]{inputenc}
\usepackage[T1]{fontenc}
\usepackage{graphicx}
\usepackage{grffile}
\usepackage{longtable}
\usepackage{wrapfig}
\usepackage{rotating}
\usepackage[normalem]{ulem}
\usepackage{amsmath}
\usepackage{textcomp}
\usepackage{amssymb}
\usepackage{capt-of}
\usepackage{todonotes}
\usepackage{fullpage}
% remove links'styles
\usepackage{hyperref}
\hypersetup{pdfborder = 0 0 0}

\author{Candidate: Laura Viglioni \\ Supervisor: Prof. Dr. Ricardo Dahab}
\date{March 12, 2021}
\title{Master’s Qualification Exam}
\hypersetup{
 pdfauthor={Candidate: Laura Viglioni \\ Supervisor: Prof. Dr. Ricardo Dahab},
 pdftitle={Master’s Qualification Exam},
 pdfkeywords={},
 pdfsubject={},
 pdfcreator={Emacs 27.1 (Org mode 9.4.4)}, 
 pdflang={English}}
\begin{document}

\maketitle
\begin{abstract}
lorem ipsum
\end{abstract}
\pagebreak


\section{Introduction}
\label{sec:org6f1c4a6}
escrever sobre como é usado em cripto, qual é nossa motivaçao, lattices sao quentes, problemas de eficiencia, segurança, alternativa nova
\subsubsection*{Motivation}
\label{sec:orgaba916c}
\subsubsection*{Summary of objectives}
\label{sec:orgcefd833}
\subsubsection*{Organization of this document}
\label{sec:orgc555a2f}
\section{Mathematical background}
\label{sec:orgc3ff093}
\subsection{Preliminaries}
\label{sec:orgc4ed3b9}
In this text we will consider the Natural Numbers \(\N\) the set of all positive integers: \(\N = \{1,2,3,\dots\}\) and \(\P\) the set of all prime numbers. 
\subsection{Groups}
\label{sec:orgb1ea44f}

\begin{definition}
  A \textbf{group} is a set $G$ closed under a binary operation $\cdot$ defined on $G$ such
  that:
  \begin{itemize}
  \item \textbf{Associativity: } $\forall a,b,c \in G, \; a\cdot(b\cdot c) = (a\cdot b)\cdot c$
  \item \textbf{Identity element: } $\exists e \in G \; ; \; \forall a \in G, \; a\cdot e = e\cdot a = a$
  \item \textbf{Inverse element: } $\forall a \in G, \; \exists b \in G \; ; \; a\cdot b = b \cdot a = e$
  \end{itemize}
And it is denoted by $\langle G,\cdot\rangle$, or simply $G$ if the operation is implied.
\end{definition}

\begin{definition}
  A group is said to be \textbf{commutative} or \textbf{abelian}
  if $\forall a, b \in G, \; a\cdot b = b\cdot a$
\end{definition}

\noindent
A group is called \textbf{additive} or \textbf{multiplicative} if its
operation is addition or multiplication, respectively.

\begin{definition}
  A subset $H$ of $G$ is a \textbf{subgroup} of $\langle G,\cdot \rangle$ if it is
  closed under $\cdot$ induced by $\langle G,\cdot \rangle$. The \textbf{trivial subgroup} of any
  group is the set consisting of just the identity element.
\end{definition}

\begin{definition}
  The \textbf{order} of a group $\langle G,\cdot\rangle$ is the cardinality of the set $G$.
\end{definition}

\begin{definition}
  A subgroup $H$ of $G$ can be used to decompose $G$ in uniform sized and
  disjoints subsets called \textbf{cosets}. Given an element $g \in G$:
  \begin{itemize}
  \item A \textbf{left coset} is defined by $gH := \{g\cdot h \; ; \; h \in H\}$
  \item A \textbf{right coset} is defined by $Hg := \{h\cdot g \; ; \; h \in H\}$
  \end{itemize}
\end{definition}

\subsection{Rings and fields}
\label{sec:orgd3a12e0}

   \begin{definition}
  A \textbf{ring} is a set together with two binary operations, we will note by
  $+$ and $*$ and call it addition and multiplication, respectively, such that:
  \begin{itemize}
  \item $\langle R,+\rangle$ is an abelian group.
  \item $*$ is associative
  \item $*$ is distributive over $+$
  \end{itemize}

  And it is denoted by $\langle R,+,*\rangle$, or simply $G$ if the operations are implied.
\end{definition}

\begin{definition}
  A ring is said to be \textbf{commutative} if its $*$ operation is commutative.
\end{definition}

\begin{definition}
  A ring is said to be \textbf{with unity} if $*$ has a identity element. We
  shall note it by $1$ and it is called \textbf{unity}.

\end{definition}

\begin{definition}
  A \textbf{division ring} is a ring R where $\forall r \in R, \; \exists s \in R \; ; \; r*s = 1$.
\end{definition}

\begin{definition}
  A \textbf{field} is a commutative division ring.
\end{definition}

\subsection{Number  fields}
\label{sec:orgf7c7547}

   \begin{definition}
  Let $K$ and $L$ be two fields, $L$ is said to be a \textbf{field extension} of
  $K$ if $L \subseteq K$ and we denote it by $L/K$
\end{definition}

Note that in a field extension \(L/K\), \(L\) has a structure of a vector space over
\(K\), where vector addition is in \(L\) and scalar multiplication \(a \in K, \; v \in L
   \; \implies av \in L\). The dimension of \(L\) as a vector space is called
\textbf{degree} and it is denoted by \([L:K]\).

\begin{definition}
  A field extension is called \textbf{number field} when it is over $\Q$.
\end{definition}

\begin{definition}
  Let $\alpha \in L$ where $L/K$ is a field extension. We say that $\alpha$ is
  \textbf{algebraic over $K$} if $\exists p \in K[X] \;;\; p(\alpha) = 0$. $p$ is said to be
  \textbf{the minimal polynomial of $\alpha$ over $K$} denoted by $p_\alpha$. If $\alpha \in L =
  \Q[\theta]$, we simply call $\alpha$ an \textbf{algebraic number}.
\end{definition}

\begin{example}
  It is known that $\Q$ is a field. If we add $\sqrt{2}$ to the set, we
  can build a new field adding also all the powers and multiples of
  $\Q$. This new field is denoted by $\Q[\sqrt{2}]$, note that
  $\sqrt{2}$ is algebraic and its minimal polynomial $p_{\sqrt{2}} = x^2-2$. All
  elements of $\Q[\sqrt{2}]$ are in the form $\{a+b\sqrt{2} \;|\; a,b \in
  \Q\}$ and one of its basis is $\{1, \sqrt{2}\}$, so it has degree is
  $2$.
\end{example}

\begin{example}
  If we add $\sqrt[3]{2}$ to $\Q$ instead, its elements would have the
  form $\{a + b\sqrt[3]{2} + c\sqrt[3]{4} \;|\; a,b,c \in \Q\}$, so one of
  its basis is $\{1 ,\sqrt[3]{2} ,\sqrt[3]{4}\}$, $p_\alpha = x^3 - 2$ and its degree
  is $3$.
\end{example}

\begin{example}[\cite{Ortiz2021}, Cyclotomic number field]
  A number field of particular interest is $\Q(\zeta_m)$, the m-th cyclotomic field,
  where $\zeta_m = \exp{2\pi i /m}$ is a primitive $m$-th root of unity for any
  integer number $m \geq 1$. The degree of $\Q(\zeta_m)$ is $\phi(m)$, where $\phi(\cdot)$
  denotes the Euler’s totient function. The minimal polynomial of $\zeta_m$, called
  the $m$-th cyclotomic polynomial, is $\Phi_m(x) = \prod_{k \in \Z_{m}^*}$, where $\Z^*_m$ denotes the group of invertible elements in $\Z/m\Z$.
\end{example}

\begin{example}[\cite{Ortiz2021}, Maximal real subfield]
  \label{ex:maximum-real-subfield}
  The number field $\Q(\zeta_m + \zeta_m^{-1}) \subset \R \cap \Q(\zeta_m)$ is the maximal real subfield of $\Q(\zeta_m)$ and has degree $\phi(m)/2$ if $m \geq 3$.
\end{example}

\begin{theorem}
  [\cite{stewart2002}, p.40] If $K$ is a number field, then $K = \Q[\theta]$ for some
  algebraic number $\theta \in K$, called primitive element.
\end{theorem}

Then we conclude that \(\{1, \theta, \theta^2, ... , \theta^{n-1}\}\) is a basis for the vector
space \(K = \Q[\theta]\) over \(\Q\). Note that we can represent an number \(a \in K\) as a linear combination of \(\theta\), \emph{i.e} \(a = \sum^n_{i=0}{a_i\theta^i}\) or as a polynomial \(a(x) = \sum^n_{i=0}{a_ix^i}\).

\begin{definition}
A number $\alpha$ is said to be an \textbf{algebraic integer} if $ p \in \Z[X] \;;\; p(\alpha) = 0$. The set of all algebraic integers of $K$ forms a ring called \textbf{ring of integers} of $K$ and is denoted by $\Ok$.
\end{definition}

\begin{definition}
An \textbf{integral basis} is a basis for a ring of integers. 
\end{definition}

\begin{definition}[\cite{Peikert2017}, Section 2.3.2]
  An \textbf{integral Ideal} $\Id \subset \Ok$ is a  nontrivial additive subgroup that
  is also closed under multiplication by $\Ok$, \textit{i.e.}, $r \cdot a \in \Id$ for
  any $r \in \Ok$ and $a \in \Id$. Any ideal $\Id$ is a free $\Z$-module of rank
  $n$, \ie, it is the set off all $\Z$-linear combinations of some basis
  $\{b_1,\dots,b_n\} \subset \Id$  of linearly independents (over $\Z$) elements $b_i$.
\end{definition}

\begin{definition}[\cite{Peikert2017}, Section 2.3.2]
  A \textbf{fractional ideal} $\Id \subset K$ is a set such that $d\Id \subset \Ok$ is an
  integral ideal for some $d \in \Ok$
\end{definition}

\begin{definition}[\cite{Peikert2017}, Section 2.3.3]
  For any fractional ideal $\Id \subset K$, its \textbf{dual ideal} is defined as
  $\Id^v \defsym \{ a \in K \;;\; Tr(a\Id) \subset \Z \}$. An important canonical
  fractional ideal in a number field K is the \textbf{codifferent ideal}
  $\Ok^v$, \ie, the dual ideal of the ring of integers: $\Ok^v \defsym \{ a \in K \;;\; Tr(a\Id) \subset \Ok \}$.
\end{definition}

    \begin{definition}[Fixed field by involution]
\label{def:fixed-field-by-involution}
      A map $f: K \rightarrow K$, where $K$ is a number field, is called \textbf{involution}
      of $K$ if $\forall a,b \in K \; f(a+b) = f(a) + f(b) \; f(a \cdot b) = f(a) \cdot f(b)$ and
      $f(f(a)) = a$. The subfield $F = \{a \in K \; f(a) = a\}$ is called \textbf{fixed field by
        involution} of $K$.
    \end{definition}
\subsection{The inner product space \emph{H}}
\label{sec:org8f4e4a1}
\begin{definition}
  \label{def:h-space}
  Let $r,s,n \in \Z_+$ such that $n = r + 2s > 0$. The space $H \subset \C^n$ is defined
  as:
  \begin{equation*}
    H = \{(a_1,\dots, a_r, b_1,\dots, b_s, \overline{b_1}, \dots, \overline{b_s}) \in \C^n\}
  \end{equation*}
  where $a_i \in \R, \; \forall i \in \{1,\dots,r\}$ and $b_j \in \C \setminus \R, \; j \in \{1,\dots,
  s\}$. For all $x = \left(x_1, \dots, x_n\right), y = \left(y_1, \dots, y_n\right) \in H$ the space
  $H$ is endowed with inner product $\langle {x,y} \rangle_H$ defined as:
  \begin{equation*}
    \langle {x,y} \rangle_H = \sum_{i=1}^n{x_i \overline{y_i}} = \sum_{i=1}^r{x_i \overline{y_i}} + \sum_{i=1}^s{x_{i+r} \overline{y_{i+r}}} + \sum_{i=1}^s{\overline{x_{i+r}} y_{i+r}}
  \end{equation*}

  The $\ell_2$-norm and infinity norm of any $x \in H$ are defined as $\|x\| =
  \sqrt{\langle{x,x}\rangle_H}$ and $\|x\|_\infty = \max{\{ |x_i| \}}_{i=1}^n $.
\end{definition}

It can be proven that \(H\) and \(\R^n\) are isomorphic.
\subsection{Lattices}
\label{sec:org58b9d00}
\subsubsection{Basic definitions}
\label{sec:org34adb07}

   \begin{definition}
  A Lattice $\Lambda \subset \R^n$ is a subgroup of the additive group $\R^n$
\end{definition}

In other words, given \(m\) linear independent vectors in \(\R^n\), the set
\(\{v_1, v_2, ..., v_m\}\) is called a \textbf{basis} for \(\Lambda\) and the Lattice may defined
by:

\begin{definition}
  \begin{equation*}
    \Lambda := \left\{x = \sum_{i=1}^m{\lambda_iv_i} \in \R^n \; | \; \lambda_i \in \Z\right\}
  \end{equation*}

\emph{I.e.}, any $\lambda \in \Lambda$ can be written as $\lambda = Mv$ where $M$ is the
\textbf{generator matrix} of $\Lambda$ where each row is a vector from the basis and
$v \in \Z^n$.
\end{definition}

Since the space \(H\) (\ref{def:h-space}) is isomorphic to \(\R^n\), all definitions above can be switched from \(\R^n\) to \(H\) without any loss of generality.

\begin{definition}
  The \textbf{minimum distance} of an Lattice $\Lambda$ is the shortest nonzero vector
  from $\Lambda$, given some norm, \textit{i.e.}:
  $$
  \lambda_1(\Lambda) \defsym \min_{0 \ne v \in \Lambda}{\|v\|}
  $$

  We define $\lambda_m$ as the set of $m \in \N$  linear independent vectors of $\Lambda$
  such that the biggest vector from $\lambda_m$ is equal or smaller than the biggest
  vector of any linear independent set of length $m$ in $\Lambda$. We usually use
  $\lambda_n$, where $n$ is the size of the basis of $\Lambda$ and we call them
  \textbf{shortest independent vectors} of $\Lambda$.
\end{definition}

\begin{definition}\label{def:gram-matrix}
  Let $\Lambda$ be a lattice and $M$ its generator matrix. The matrix $G = MM^T$ is called \textbf{Gram matrix} for $\Lambda$.
\end{definition}
\subsubsection{Lattice problems}
\label{sec:org3a326ab}
\begin{definition}[\cite{Peikert2017}, Definition 2.8, Gap Shortest Vector
  Problem]
  \label{def:gapsvp}
For an approximation factor $\gamma  = \gamma(n) \geq 1$, the $GapSVP_\gamma $ is: given a lattice
$\Lambda$ and length $d > 0$, output \textbf{YES} if $\lambda_1(\Lambda) \leq d$ and \textbf{NO} if
$\lambda_1(L) > \gamma d$.  
\end{definition}

\begin{definition}[\cite{Peikert2017}, Definition 2.8, Shortest Independent
  Vectors Problem]
  \label{def:sivp}
  For an approximation factor $\gamma = \gamma(n) \geq 1$, the $SIVP_\gamma$ is: given a lattice $\Lambda$, output $n$ linearly independent lattice vectors of length at most $\gamma(n) \cdot \lambda_n(\Lambda)$.
\end{definition}

\subsection{Learning problems}
\label{sec:org40422cd}
I this section we will describe some problems that are believed to be hard and used in cryptography. 
\subsubsection{Learning from Parity}
\label{sec:orgce8af1a}
 \begin{definition}
  \label{def:LWP}
  Given $m$ vectors uniformly chosen  $a_i \gets \Z^n_2$ and some $\epsilon \in [0,1]$, we
  define the problem \textbf{Learn With Parity (LWP)} as:

  find $s \in \Z^n_2$ such that for $i \in \{1,\dots,m\}$
     $$ \langle{s, a_i}\rangle \; \approx_\epsilon \; b_i \;\; (mod\; 2) $$

     In other words, the equality holds with probability $1 - \epsilon$

\end{definition}

\subsubsection{Learning with Errors}
\label{sec:orgc27dbe6}
\begin{definition}
  \label{def:LWE}
  Learning With Errors (LWE) is a generalization of LFP (\ref{def:LWP}) with two new parameters $p \in \P$ and $\chi$ a probability distribution on $\Z_p$ so that we have:

     $$ <s, a_i> \; \approx_\chi \; b_i \;\; (mod\; p) $$
     or
     $$ <s, a_i> + e_i \; = \;  b_i \;\; (mod\; p) $$

     Where $a_i \gets \Z^n_p$ uniformly and $e_i \gets \Z$ according to $\chi$

\end{definition}

\begin{theorem}[\cite{regev2009}, Theorem 1.1]
  Let $n$, $p$ be integers and $\alpha \in (0, 1)$ be such that $\alpha p > 2\sqrt{n}$. If
  there exists an efficient algorithm that solves $LWE_{p \Psi_\alpha}$ then there
  exists an efficient quantum algorithm that approximates the decision version
  of the shortest vector problem ($GAP_{SVP}$~\ref{def:gapsvp}) and the
  shortest independent vectors problem (SIVP~\ref{def:sivp}) to within
  $\tilde{O}(n/\alpha)$ in the worst case.

  Where $\Psi_\beta$ is defined as:
  $$
  \forall r \in [0,1), \; \Psi_\beta(r) \defsym \sum_{k=-\infty}^\infty{\frac{1}{\beta} . \exp{\left( -\pi \left( \frac{r-k}{\beta} \right)^2 \right)}}
  $$
\end{theorem}

\subsubsection{Ring-LWE}
\label{sec:orgea55b39}
\begin{text}
  Let $K$ be a number field, $R = \Ok$ its ring of integers and $R^\vee$ the
  codifferent ideal of $K$. Let $2 \leq q \in \N$ and for any fractional ideal $\Id \subset
  K$. Also let $K_\R$ be the tensor product $K \otimes_\Q \R$, $\Id_q = \Id/q\Id$
  and $\mathbb{T} = K_\R/R^\vee$.

  The twisted embeddings can be extended from $K$ to $K_\R$ as follows [\cite{Ortiz2021},
  Section 3]: for any totally positive $\tau \in F$, the $\R$-vector space
  $\sigma_\tau(K_\R)$ is isomorphic to $H \simeq \R^n$. Consider the extension of the trace
  function $Tr_K : K \rightarrow \Q$ to $Tr_K : K_\R \rightarrow \R$, for any $\tau \in F$ totally
  positive integer we can define the inner product as:

  $$
  \langle{a,b}\rangle_\tau \defsym \langle{\sigma_\tau(a), \sigma_\tau(b)}\rangle_H  = Tr_K (\tau a \overline{b}) , \;\; a,b \in K_\R
  $$

  By considering the inner product $\langle{a,b}\rangle_\tau$, the $\R$-vector space $K_\R$
  is an Euclidian vector space of dimention $n$ isometric to both $\left(
    H , \langle{a,b}\rangle_H  \right)$ and $\left( \R , \langle{a,b}\rangle  \right)$.
\end{text}

\begin{definition}[\cite{Peikert2017}, Definition 2.15, Ring-LWE Average-Case Decision]
  \label{def:rlwe-decision}
  Let $\Upsilon$ be a distribution over a family of error distributions over $K_\R$.
  The average-case Ring-LWE decision problem, denoted $R-LWE{q,\Upsilon}$, is to
  distinguish (with non-negligible advantage) between independent samples from
  $A_{s, \psi}$ for a \textit{random} choice of $(s,\psi) \longleftarrow U(R_q^\vee) \times \Upsilon$, and the
  same number of uniformly random and independent samples from $R_q \times \mathbb{T}$.
\end{definition}

\begin{theorem}[\cite{Peikert2017}, Corollary 5.2]
  Let $\alpha = \alpha(n) \in (0, 1)$, and let $q = q(n)$ be an integer such that $\alpha q \geq 
  2\sqrt{n}$. Then, there is \emph{a polynomial-time quantum reduction from} $SIVP_{\gamma'}$
  and $GapSVP_{\gamma'}$ \emph{to (average-case, decision)} $LWE_{q,\alpha}$.
\end{theorem}

\begin{definition}[\cite{Lyubashevsky2010}, Definition 3.2, Ring-LWE Search]
  \label{def:rlwe-search}
Let $\Psi$ be a family of distributions over $K_\R$. The search version of the $ring-LWE$ problem, denoted $R-LWE_{q,\Psi}$, is defined as follows: given access to arbitrarily many independent samples from $A_{s,\psi}$ for some arbitrary $s \in R_q^\vee$ and $\psi \in \Psi$, find $s$.
\end{definition}

\begin{theorem}[\cite{Lyubashevsky2010}, Theorem 3.6]
  Let K be the mth cyclotomic number field having dimension $n = \phi(m)$ and $R =
  \Ok$ be its ring of integers. Let $\alpha < \sqrt{(\log{n})/n}$, and let $q = q(n)
  \geq 2, \; q = 1 \; (mod \; m)$ be a $poly(n)$-bounded prime such that $\alpha q \geq
  \omega(\sqrt{\log{n}})$. Then there is a polynomial-time quantum reduction from
  $\tilde{O}(n/\alpha)$-approximate $SIVP$ (or $SVP$) on ideal lattices in $K$ to
  $R-DLWE_{q,\Upsilon_\alpha}$. Alternatively, for any $l \geq 1$, we can replace the target
  problem by the problem of solving $R-DLWE_{q,D_\xi}$ given only $l$ samples,
  where $\xi = \alpha \cdot ( nl/ \log{(nl)} )^{1/4}$
\end{theorem}

\subsection{Twisted Embeddings}
\label{sec:org88320a4}
\subsubsection{Embeddings}
\label{sec:org50b0c0b}

\begin{definition}
Let $K$ and $L$ be two field extensions and a homomorphism $\phi: K \rightarrow L$. $\phi$ is said to be a \textbf{$\Q$-homomorphism} if $\phi(a) = a, ; \forall a \in \Q$ 
\end{definition}

\begin{definition}
A $\Q-homomorphism ; \phi: K \rightarrow \C$ is called an \textbf{embedding}.
\end{definition}

\begin{theorem}
[\cite{stewart2002}, p.41] If $K$ is a number field with degree $n$ then there are
exactly $n$ embeddings $\sigma_i : K \rightarrow \C$ where by $\sigma_i(\theta) =
\theta_i$ where $\theta_i \in \C$ is a distinct zero of the $K$'s
minimum polynomial.
\end{theorem}

      \begin{definition}[Trace and Norm]
  \label{def:trace-and-norm}
  Let $x \in K$ be an element of a number field and $\{\sigma_i\}_{i=1}^n$ the possible
  embeddings. The elements $\{\sigma_i(x)\}_{i=1}^n$ are called \textbf{conjugates} of
  x and we define the \textbf{norm} of $x$ $N(x)$ and \textbf{Trace} of $x$ $Tr(x)$
  respectively:
  $$
  N(x) = \prod_{i=1}^n{\sigma_i(x)} \;,\;   Tr(x) = \sum_{i=1}^n{\sigma_i(x)}
  $$

\end{definition}
\begin{theorem}[\cite{stewart2002}, p.54]
  For any $x \in K$, we have $N(x), Tr(x) \in \Q$. If $x \in \Ok$, we have $N(x),
  Tr(x) \in \Z$.
\end{theorem}


   \begin{definition}
Let $\{\sigma_i\}_n$ the possible embeddings of a number field $K$. Let $r$ the number of embeddings with real images and $2s$ the complex ones, then
$r + 2s = n$. The pair $\left(r,s\right)$ is called \textbf{signature} of $K$.
\end{definition}

   \begin{definition}\label{def:canonical-embedding}
The homomorphism $\sigma: K \rightarrow \R^r \times \C^s$, where $(r,s)$ is the signature of $K$, is
said to be the \textbf{canonical embedding} and is defined by:
$$
\sigma(x) = \left(\sigma_1(x), ... , \sigma_r(x), \sigma_{r+1}(x), ..., \sigma_{r+s}(x) \right)
$$

Note that we could rewrite the canonical embedding as $\sigma : K \rightarrow \R^n$
$$
\sigma(x) = \left( \sigma_1(x), ... , \sigma_r(x), \Re(\sigma_{r+1}(x)), \Im(\sigma_{r+1}(x)), ...,
  \Re(\sigma_{r+s}(x)), \Im(\sigma_{r+s}(x)) \right)
$$

For now on we will denote it simply by:

$$
\sigma(x) = \left( \sigma_1(x), \dots , \sigma_r(x), \sigma_{r+1}(x), \dots, \sigma_{r+2s}(x) \right)
$$

\end{definition}

\subsubsection{Algebraic lattices}
\label{sec:org243fb52}

\begin{theorem}[\cite{stewart2002}, p.155]\label{theo:algebraic-lattice}
Let $\{\omega_1,...,\omega_n\}$ be an integral basis of $K$, The $n$ vectors $v_i = \sigma(\omega_i)
\in \R^n$ are linearly independent, so they define a full rank algebraic lattice
$\Lambda = \Lambda(\Ok) = \sigma(\Ok)$.
\end{theorem} 
The generator matrix of \(\Lambda = \sigma(\Ok)\) is defined by:

\begin{equation}
  \label{def:gen-matrix-alg-lattices}
  \begin{pmatrix}
    \sigma_1(\omega_1) & $\dots$ &  \sigma_{r+2s}(\omega_1) \\
    & \vdots & \\
    \sigma_1(\omega_n) & $\dots$ & \sigma_{r+2s}(\omega_n) \\
  \end{pmatrix}  
\end{equation}

\begin{remark}\label{rem:lat-int-correspondence}
  An embedding creates the correspondence between a point $\lambda \in \Lambda \subset \R^n$ of an algebraic lattice (Theo.
 ~\ref{theo:algebraic-lattice}) and an integer in $\Ok$:

  Let $\lambda$ be a point of a lattice $\Lambda$:

\begin{align*} 
     \lambda &= (\lambda_1,\dots,\lambda_{r+2s}) \in \Lambda \\
       &= \left( \sum_{i=1}^n{z_i\sigma_1(\omega_i)} , \dots , \sum_{i=1}^n{z_i\sigma_{r+2s}(\omega_i)} \right) \\
       &= \left( \sigma_1\left(   \sum_{i=1}^n{z_i\omega_i} \right) , \dots , \sigma_{r+2s} \left( \sum_{i=1}^n{z_i\omega_i}  \right) \right) 
\end{align*}
  where $z_i \in \Z$. Since any element $x \in \Ok$ has the form $x =
  \sum_{i=1}^n{\lambda_i\omega_i}$, we can conclude that:

  \begin{equation*}
    \lambda = \left( \sigma_1(x), \dots, \sigma_{r+2s}(x) \right) = \sigma(x)
  \end{equation*}

\end{remark}

\subsubsection{Twisted embeddings}
\label{sec:org2d010e7}

\begin{definition}
  Let $K$ be a number field with degree $n$ and $\sigma$ an embedding. We say that a
  number $\tau \in F$, where $F$ is the fixed field by involution of $K$ (Definition~\ref{def:fixed-field-by-involution}) is \textbf{totally  positive} if $\forall i \in {1, \dots , n}, \; \sigma_i(\tau) \in \R^*_+$. 
\end{definition}


\begin{definition}[Twisted Embedding]
  \label{def:twisted-emb}
  Given $\tau$ a totally positive number, the \textbf{$\tau$-twisted embedding}, or
  simply twisted embedding, is the monomorphism defined as:
  \begin{equation*}
    \sigma_\tau(x) = \left( \sqrt{\tau_1}\sigma_1(x), \dots, \sqrt{\tau_{r+2s}}\sigma_{r+2s}(x) \right)
  \end{equation*}

  where $\tau_i = \sigma_i(\tau)$.
\end{definition}
\section{Twisted embeddings and cryptography}
\label{sec:org02d0130}
\subsection{Twisted Ring-LWE}
\label{sec:orgb495212}
In this section we present variant of the Ring-LWE (Definition\textasciitilde{}\ref{def:rlwe-search}) using twisted embeddings (Definition\textasciitilde{}\ref{def:twisted-emb}).



\begin{definition}[\cite{Ortiz2021}, Twisted Ring-LWE distribution]
  \label{def:twisted-ring-lwe}
  For a totally positive element $\tau \in F$, let $\psi_\tau$ denote an error distribution
  over the inner product $\langle{\cdot,\cdot}\rangle_\tau$ and $s \in R^\vee_q$ (the “secret”) be an
  uniformly randomized element. The \emph{Twisted Ring-LWE distribution}
  $\mathcal{A}_{s,\psi_\tau}$ produces samples of the form
  $$
  (a, b = a \cdot s + e \;\;\; \mod{qR^\vee}) \in R_q \times K_\R/qR^\vee.
  $$
\end{definition}

Solving the Twisted Ring-LWE is as hard as solving the usual Ring-LWE as stated in Theorem\textasciitilde{}\ref{theo:twisted-rlwe-hardness}:

\begin{theorem}[\cite{Ortiz2021}, Theorem 1]
  \label{theo:twisted-rlwe-hardness}
  Let $K$ be an arbitrary number field, and let $\tau \in F$ be totally positive.
  Also, let $(s,\psi)$ be randomly chosen from $(U(R_q^\vee)\times \Psi)$ in $(K_\R,\langle{\cdot,\cdot}\rangle_{\tau=1})$.
  Then there is a polynomial-time reduction from $Ring-LWE_{q,\psi}$ to $Ring-LWE^\tau_{q,\psi_\tau}$ .
\end{theorem}
\subsection{Error sampling in rotated \(\Z^n\)-lattices}
\label{sec:orgf2a5e1b}

\begin{text}
  In this section we present the \textit{Ortiz et al.} (\cite{Ortiz2021}, Section 8)
  variation of the cryptosystem of Lyubashevsky, Peikert, and Regev
  (\cite{LPV2013}, Section 8.2) using twisted embeddings. Let $R$ be an $m$-th
  cyclotomic ring and $p, q \in \Z$ coprimes. The message space is defined as
  $R_p$ and it is required $q$ to be coprime with every odd prime dividing
  $m$. Consider that $\phi_\tau$ is an error distribution over $\krspace$
  and $\lfloor{\cdot}\rceil$ denotes a valid discretization to (cosets) of $R^\vee$ or $pR^\vee$.
  Also, $\hat{m} = m/2$ if $m$ is even, otherwise $\hat{m} = m$. Finally, for any
  $\overline{a} \in \Z_q$, let $[[\overline{a}]]$ denote the unique representative
  $a \in (\overline{a} + q\Z) \cap [-q/2, q/2)$, which is entry-wise extended to
  polynomials.

  \begin{itemize}
  \item \textbf{Key generation}: choose a uniformly random $a \in R_q$. Choose $x
    \longleftarrow \lfloor{\phi_\tau}\rceil$ and $e \longleftarrow \lfloor{p \cdot \phi_\tau}\rceil_{pR^\vee}$. Output $(a,b = \hat{m}\cdot(a \cdot x + e)
    \mod{qR} ) \in R_q \times R_q$ as the public key and $x$ as the secret key.
  \item \textbf{Encryption}: choose $z \longleftarrow \longleftarrow \lfloor{\phi_\tau}\rceil_R^\vee$, $e' \longleftarrow \lfloor{p \cdot
      \phi_\tau}\rceil_{pR^\vee}$ and  $e'' \longleftarrow \lfloor{p \cdot \phi_\tau}\rceil_{t^{-1}\mu +pR^\vee}$, where $\mu \in R_p$ is
    the word to be encrypted. Let $u = \hat{m} \cdot (a \cdot z + e') \mod{qR}$ and $v =
    z \cdot b + e'' \in R_q^\vee$. Output $(u,v) \in R_q \times R^\vee_q$.
  \item \textbf{Decryption}: Given the encrypted message $(u,v)$, compute $v - u
    \cdot x \mod{qR^\vee}$, and decode it to $d = [[v - u \cdot x]] \in R^\vee$. Output $\mu = t \cdot
    d \mod{pR}$. 
  \end{itemize}

  In this cryptosystem, the most expensive operations to compute are the error
  sampling, its discretization and the polynomial multiplications. When $R$ is
  the ring of integers of the maximum real subfield
  (\ref{ex:maximum-real-subfield}) $\maxrs$, the sampling of error
  terms can be performed directly over $(K_\R, \langle{\cdot,\cdot}\rangle_\tau)$ in the orthonormal
  basis while preserving the spherical format and standard deviation in respect
  to the corresponding distribution in $H$. The efficiency
of discrete sampling when $K = \Q(\zeta_p + \zeta_p^{-1})$ is reinforced by the fact
that the discretization in $\Z^n$-lattices is simply a coordinate-wise rounding to the nearest integer. (\cite{Ortiz2021}, Section 8).
\end{text}
\subsection{Impacts of the twisted embeddings}
\label{sec:org956960d}

\begin{text}
  The correspondence between a point $\lambda \in \Lambda$ of a lattice and an algebraic
  integer $x \in \Ok$ of a ring of integers (Remark~\ref{rem:lat-int-correspondence}),
  \ie, $\lambda = (\sigma_1(x), \dots, \sigma_{r+2s}(x)) = \sigma(x)$, where $\sigma$ is the
  canonical embedding (Definition~\ref{def:canonical-embedding}), allow us to
  sample errors over a Lattice and convert them through the embedding to the
  polynomial representation, \ie, the representation of an element of a ring of
  integers.

  This conversion is trivial when the Lattices we are dealing are rotations of
  $\Z^n$, otherwise it can be very expensive. With the canonical embedding
  (Definition~\ref{def:canonical-embedding}) we can achieve a $\Z^n$ rotated
  Lattice with the cyclotomic number field with power of $2$ dimension
  (\cite{Lyubashevsky2010},~\cite{DucasDurmos2012}).

  Using the Twisted Embedding (Definition~\ref{def:twisted-emb}) we can obtain
  different lattices from the same number field:

\end{text}

   \begin{example}[\cite{Ortiz2021}, Example 3]
  Let $K = \Q(\sqrt{3}) = \{a + b\sqrt{3} \;;\; a,b \in \Q\}$ be a totally
  real number field with degree 2. It follows that the fixed field by
  involution $F=K$. For any totally positive element $\tau \in F$, consider
  the lattice $M_\tau = \Ok = \Z[\sqrt{3}]$ in the inner product space
  $(K_\R,\langle \cdot,\cdot \rangle_\tau)$. The set $\{1,\sqrt{3}\}$ in a
  $\Z$-basis of $M_\tau$ and the Gram matrix of the lattice $M_\tau$ is given by:
  \[G_\tau =
    \begin{bmatrix}
      Tr_K(\tau) & Tr_K(\tau\sqrt{3}) \\
      Tr_K(\tau\sqrt{3}) & Tr_k(3\tau)
    \end{bmatrix}
  \]

  For example, for $\tau = 1$ \text{and} $\tau = 2 + \sqrt{3}$, the Gram matrices are
  given by:
  \[
    G_1 =
    \begin{bmatrix}
      2 & 0 \\
      0 & 6
    \end{bmatrix}
    \;\;\;and\;\;\;
    G_{2+\sqrt{3}} =
    \begin{bmatrix}
      4 & 6 \\
      6 & 12
    \end{bmatrix}
  \]
  It can be shown that these two lattices are not equivalent.
\end{example}

\begin{text}
  The theorem (Theorem~\ref{theo-ideal-lattices-doesnt-change-gaussian}),
  proposition
  (Proposition~\ref{prop:maximal-real-subfield-generates-orthonormal-lattice})
  and corollary (Corollary~\ref{corollary:maximal-real-subfield-prime-p-greater-than-5}) bellow show that we can build $\Z^n$-rotated
  lattices from the maximal real subfield
  (Example~\ref{ex:maximum-real-subfield}) using twisted embeddings, \ie, the
  errors sampled on these lattices can be trivially converted to polynomial
  representation as elements of a number field.
\end{text}

\begin{theorem}[\cite{Ortiz2021}, Theorem 5]\label{theo-ideal-lattices-doesnt-change-gaussian}
  Let $K$ be a number field with fixed field by the involution $F$. Consider $\tau
  \in F$ totally positive and $\Id \subset \Ok$ a fractional ideal such that $\Id$ is an
  ideal lattice in $(K_\R , \langle \cdot,\cdot \rangle_\tau )$. If $\Id$ is an orthonormal lattice, then
  both the format and the standard deviation of a spherical Gaussian
  distribution in an orthonormal basis of $\Id \subset K_\R$ are preserved when seen in
  the canonical basis of the space $H$ (via the twisted embedding $\sigma_\tau$).
\end{theorem}

\begin{proposition}[\cite{Ortiz2021}, Proposition 2]\label{prop:maximal-real-subfield-generates-orthonormal-lattice}
Let $p \geq 5$ be a prime number, and let $K = \maxrs$ and $\tau =
  \frac{1}{p}(1 - \zeta_p)( 1 - \zeta^{- 1}_p)$. Then $\Ok$ in $\krspace$ is an
  orthonormal lattice with basis $\Cb^\perp = \{e_1^\prime, \dots, e^\prime_n \;;\; e^\prime_n = e_n
  \;\; \text{and} \;\; e^\prime_j = e_j +  e^\prime_{j+ 1} \}$ where $\Cb = \{e_1,\dots,e_n\}$
  is the integral basis of $K$.
\end{proposition}

\begin{corollary}[\cite{Ortiz2021}, Corollary 1]\label{corollary:maximal-real-subfield-prime-p-greater-than-5}
  Let $K = \maxrs$ for $p \geq 5$ prime and let $v \in \Ok$ be a random variable
  distributed as $\psi_s^n$ in the basis $\Cb^\perp$. Then, the dstribution of $(T^{-1}
  \circ \sigma_\tau)(v)$ for $\tau = \frac{1}{p}(1 - \zeta_p)( 1 - \zeta^{- 1}_p)$, seen in the
  canonical basis of $H$, is the spherical Gaussian $\psi_s^n$.
\end{corollary}

These new constructions with a more variety of possible rings increase the security notions of Ring-LWE (Definitions\textasciitilde{}\ref{def:rlwe-search},\textasciitilde{}\ref{def:rlwe-decision}), since specific rings might have specific vulnerabilities, thinking about cryptosystems security, that other rings don't. It's important to remark that each number field has it's own polynomial representation and specifically a polynomial \(f(x)\) that defines the ring we use as a parameter in the Ring-LWE cryptosystems. That said, the size of the parameters, therefore keys, encrypted messages etc, and the cost of the Ring-LWE operations depends on the polynomial representation of the ring and of \(f(x)\).

There is though an open question if there exist other number fields that we build orthonormal lattices and its polynomial arithmetic are efficiently enough for be used in cryptosystems.

\section{Objectives}
\label{sec:orgd57af41}
validar a ideia de twisted embedings em varios aspectos, investigaçao em parte teorica e pratica das hipoteses levantadas no artigo sobre as vantagens de usar o twisted, practical impacts do artigo
\section{Methodology}
\label{sec:org5dc7f08}
\subsection{Literature review}
\label{sec:org7e06953}
\subsection{Activities}
\label{sec:org5e78b2b}
\begin{itemize}
\item Second semester of 2021
\item \ldots{}
\end{itemize}
\section{Conclusion}
\label{sec:orgb18dbdd}

\bibliographystyle{plain}
\bibliography{library,ic-tese-v3}
\end{document}