% Created 2021-01-25 Mon 19:02
\documentclass[Ingles]{ic-tese-v3}
\usepackage[utf8]{inputenc}
\usepackage[T1]{fontenc}
\usepackage{graphicx}
\usepackage{grffile}
\usepackage{longtable}
\usepackage{wrapfig}
\usepackage{rotating}
\usepackage[normalem]{ulem}
\usepackage{amsmath}
\usepackage{textcomp}
\usepackage{amssymb}
\usepackage{capt-of}
\usepackage{hyperref}
%
% American mathematical society libs
%

% imports
\usepackage{amsmath,amssymb,latexsym,amsfonts,amsthm, mathtools}

% definitions
\newtheorem{theorem}{Theorem}[section]
\newtheorem{lemma}{Lemma}[section]
\newtheorem{proposition}{Proposition}[theorem]
\newtheorem{corollary}{Corollary}[theorem]

\theoremstyle{definition}
\newtheorem{definition}{Definition}[section]
\newtheorem{example}{Example}[section]
\newtheorem{remark}{Remark}[section]

%
% Commands and aliases
%

% Complex numbers set
\newcommand{\C}{\mathbb{C}}

% Real numbers set
\newcommand{\R}{\mathbb{R}}
\newcommand{\Rn}{\mathbb{R}^n}


% Rational numbers set
\newcommand{\Q}{\mathbb{Q}}

% Integer numbers set
\newcommand{\Z}{\mathbb{Z}}

% Natural numbers set
\newcommand{\N}{\mathbb{N}}

% Prime numbers set
\renewcommand{\P}{\mathbb{P}}

% Ring of integers set
\renewcommand{\O}{\mathcal{O}}
\newcommand{\Ok}{\mathcal{O}_K}

% Ring of integers set
\newcommand{\Id}{\mathfrak{I}}

% Canonical basis
\newcommand{\Cb}{\mathcal{C}}


% Definition
\newcommand{\defsym}{\vcentcolon =}

% Id Est i.e.
\newcommand{\ie}{\textit{i.e.}}

% Maximum real subfield
\newcommand{\maxrs}{\Q(\zeta_{p} + \zeta_{p} ^{-1})}

% K_\R inner space
\newcommand{\krspace}{(K_\R,\langle{\cdot,\cdot}\rangle_\tau)}
\date{}
\title{}
\hypersetup{
 pdfauthor={Laura Viglioni},
 pdftitle={},
 pdfkeywords={},
 pdfsubject={},
 pdfcreator={Emacs 27.1 (Org mode 9.4.3)}, 
 pdflang={English}}
\begin{document}

\autora{Laura Viglioni}

\title{The Dissertation or Thesis Title in English}

\orientador{Prof. Dr. Ricardo Dahab}

\mestrado

\datadadefesa{22}{04}{1500}

\paginasiniciais



\chapter{Introduction}
\label{sec:orgdf7d7a1}

\chapter{Mathematical Background}
\label{sec:orgd96b545}

\section{Groups}
\label{sec:org4cc200e}

   \begin{definition}
  A \textbf{group} is a set $G$ closed under a binary operation $\cdot$ defined on $G$ such
  that:
  \begin{itemize}
  \item \textbf{Associativity: } $\forall a,b,c \in G, \; a\cdot(b\cdot c) = (a\cdot b)\cdot c$
  \item \textbf{Identity element: } $\exists e \in G \; ; \; \forall a \in G, \; a\cdot e = e\cdot a = a$
  \item \textbf{Inverse element: } $\forall a \in G, \; \exists b \in G \; ; \; a\cdot b = b \cdot a = e$
  \end{itemize}
And it is denoted by $\langle G,\cdot\rangle$, or simply $G$ if the operation is implied.
\end{definition}

\begin{definition}
  A group is said to be \textbf{commutative} or \textbf{abelian}
  if $\forall a, b \in G, \; a\cdot b = b\cdot a$
\end{definition}

\noindent
A group is called \textbf{additive} or \textbf{multiplicative} if its
operation is addition or multiplication, respectively.

\begin{definition}
  A subset $H$ of $G$ is a \textbf{subgroup} of $\langle G,\cdot \rangle$ if it is
  closed under $\cdot$ induced by $\langle G,\cdot \rangle$.
\end{definition}

\begin{definition}
  The \textbf{order} of a group $\langle G,\cdot\rangle$ is the cardinality of the set $G$.
\end{definition}

\begin{definition}
  A subgroup $H$ of $G$ can be used to decompose $G$ in uniform sized and
  disjoints subsets called \textbf{cosets}. Given an element $g \in G$:
  \begin{itemize}
  \item A \textbf{left coset} is defined by $gH := \{g\cdot h \; ; \; h \in H\}$
  \item A \textbf{right coset} is defined by $Hg := \{h\cdot g \; ; \; h \in H\}$
  \end{itemize}
\end{definition}

\section{Rings and Fields}
\label{sec:orga2deb6b}

   \begin{definition}
  A \textbf{ring} is a set together with two binary operations, we will note by
  $+$ and $*$ and call it addition and multiplication, respectively, such that:
  \begin{itemize}
  \item $\langle R,+\rangle$ is an abelian group.
  \item $*$ is associative
  \item $*$ is distributive over $+$
  \end{itemize}

  And it is denoted by $\langle R,+,*\rangle$, or simply $G$ if the operations are implied.
\end{definition}

\begin{definition}
  A ring is said to be \textbf{commutative} if its $*$ operation is commutative.
\end{definition}

\begin{definition}
  A ring is said to be \textbf{with unity} if $*$ has a identity element. We
  shall note it by $1$ and it is called \textbf{unity}.

\end{definition}

\begin{definition}
  A \textbf{division ring} is a ring R where $\forall r \in R, \; \exists s \in R \; ; \; r*s = 1$.
\end{definition}

\begin{definition}
  A \textbf{field} is a commutative division ring.
\end{definition}

\section{Lattices}
\label{sec:org398e4c5}

   \begin{definition}
  A Lattice $\Lambda \subset \R^n$ is a subgroup of the additive group $\R^n$
\end{definition}

In other words, given \(m\) linear independent vectors in \(\R^n\), the set
\(\{v_1, v_2, ..., v_m\}\) is called a \textbf{basis} for \(\Lambda\) and the Lattice may defined
by:

\begin{definition}
  \begin{equation*}
    \Lambda := \left\{x = \sum_{i=1}^m{\lambda_iv_i} \in \R^n \; | \; \lambda_i \in \Z\right\}
  \end{equation*}

\emph{I.e.}, any $\lambda \in \Lambda$ can be written as $\lambda = Mv$ where $M$ is the
\textbf{generator matrix} of $\Lambda$ where each row is a vector from the basis and
$v \in \Z^n$.
\end{definition}

\section{Number Fields}
\label{sec:orga83df42}

   \begin{definition}
  Let $K$ and $L$ be two fields, $L$ is said to be a \textbf{field extension} of
  $K$ if $L \subseteq K$ and we denote it by $L/K$
\end{definition}

Note that in a field extension \(L/K\), \(L\) has a structure of a vector space over
\(K\), where vector addition is in \(L\) and scalar multiplication \(a \in K, \; v \in L
   \; \implies av \in L\). The dimension of \(L\) as a vector space is called
\textbf{degree} and it is denoted by \([L:K]\).

\begin{definition}
  A field extension is called \textbf{number field} when it is over $\Q$.
\end{definition}

\begin{definition}
  Let $\alpha \in L$ where $L/K$ is a field extension. We say that $\alpha$ is
  \textbf{algebraic over $K$} if $\exists p \in K[X] \;;\; p(\alpha) = 0$. $p$ is said to be
  \textbf{the minimal polynomial of $\alpha$ over $K$} denoted by $p_\alpha$. If $\alpha \in L =
  \Q[\theta]$, we simply call $\alpha$ an \textbf{algebraic number}.
\end{definition}

\begin{example}
  It is known that $\Q$ is a field. If we add $\sqrt{2}$ to the set, we
  can build a new field adding also all the powers and multiples of
  $\Q$. This new field is denoted by $\Q[\sqrt{2}]$, note that
  $\sqrt{2}$ is algebraic and its minimal polynomial $p_{\sqrt{2}} = x^2-2$. All
  elements of $\Q[\sqrt{2}]$ are in the form $\{a+b\sqrt{2} \;|\; a,b \in
  \Q\}$ and one of its basis is $\{1, \sqrt{2}\}$, so it has degree is
  $2$.
\end{example}

\begin{example}
  If we add $\sqrt[3]{2}$ to $\Q$ instead, its elements would have the
  form $\{a + b\sqrt[3]{2} + c\sqrt[3]{4} \;|\; a,b,c \in \Q\}$, so one of
  its basis is $\{1 ,\sqrt[3]{2} ,\sqrt[3]{4}\}$, $p_\alpha = x^3 - 2$ and its degree
  is $3$.
\end{example}

\begin{theorem}
  [add font 45 p.40] If $K$ is a number field, then $K = \Q[\theta]$ for some
  algebraic number $\theta \in K$, called primitive element.
\end{theorem}

Then we conclude that \(\{1, \theta, \theta^2, ... , \theta^{n-1}\}\) is a basis for the vector
space \(K = \Q[\theta]\) over \(\Q\). Note that we can represent an number \(a \in K\) as a linear combination of \(\theta\), \emph{i.e} \(a = \sum^n_{i=0}{a_i\theta^i}\) or as a polynomial \(a(x) = \sum^n_{i=0}{a_ix^i}\).

\begin{definition}
A number $\alpha$ is said to be an \textbf{algebraic integer} if $ p \in \Z[X] \;;\; p(\alpha) = 0$. The set of all algebraic integers of $K$ forms a ring called \textbf{ring of ingegers} of $K$ and is denoted by $\Ok$.
\end{definition}

\begin{definition}
An \textbf{integral basis} is a basis for a ring of integers. 
\end{definition}

\section{Twisted Embeddings}
\label{sec:org5049c90}
\subsection{Embeddings}
\label{sec:orgba0577b}

\begin{definition}
Let $K$ and $L$ be two field extensions and a homomorphism $\phi: K \rightarrow L$. $\phi$ is said to be a \textbf{$\Q$-homomorphism} if $\phi(a) = a, ; \forall a \in \Q$ 
\end{definition}

\begin{definition}
A $\Q-homomorphism ; \phi: K \rightarrow \C$ is callend an \textbf{embedding}.
\end{definition}

\begin{theorem}
[inserir fonte 45, p.41] If $K$ is a number field with degree $n$ then there are
exactly $n$ embeddings $\sigma_i : K \rightarrow \C$ where by $\sigma_i(\theta) =
\theta_i$ where $\theta_i \in \C$ is a distinct zero of the $K$'s
mininum polynomial.
\end{theorem}

   \begin{definition}
Let $\{\sigma_i\}_n$ the possible embeddings of a number field $K$. Let $r$ the number of embeddings with real images and $2s$ the complex ones, then
$r + 2s = n$. The pair $\left(r,s\right)$ is called \textbf{signature} of $K$.
\end{definition}

   \begin{definition}\label{def:canonical-embedding}
The homomorphism $\sigma: K \rightarrow \R^r \times \C^s$, where $(r,s)$ is the signature of $K$, is
said to be the \textbf{canonical embedding} and is defined by:
$$
\sigma(x) = \left(\sigma_1(x), ... , \sigma_r(x), \sigma_{r+1}(x), ..., \sigma_{r+s}(x) \right)
$$

Note that we could rewrite the canonical embedding as $\sigma : K \rightarrow \R^n$
$$
\sigma(x) = \left( \sigma_1(x), ... , \sigma_r(x), \Re(\sigma_{r+1}(x)), \Im(\sigma_{r+1}(x)), ...,
  \Re(\sigma_{r+s}(x)), \Im(\sigma_{r+s}(x)) \right)
$$

For now on we will denote it simply by:

$$
\sigma(x) = \left( \sigma_1(x), \dots , \sigma_r(x), \sigma_{r+1}(x), \dots, \sigma_{r+2s}(x) \right)
$$

\end{definition}

\subsection{Algebraic Lattices}
\label{sec:org2853cc6}

\begin{theorem}[adicionar citaçao 45, p155]\label{theo:algebraic-lattice}
Let $\{\omega_1,...,\omega_n\}$ be an integral basis of $K$, The $n$ vectors $v_i = \sigma(\omega_i)
\in \R^n$ are linearly independent, so thety define a full rank algebraic lattice
$\Lambda = \Lambda(\Ok) = \sigma(\Ok)$.
\end{theorem} 
The generator matrix of \(\Lambda = \sigma(\Ok)\) is defined by:

\begin{equation}
  \label{def:gen-matrix-alg-lattices}
  \begin{pmatrix}
    \sigma_1(\omega_1) & $\dots$ &  \sigma_{r+2s}(\omega_1) \\
    & \vdots & \\
    \sigma_1(\omega_n) & $\dots$ & \sigma_{r+2s}(\omega_n) \\
  \end{pmatrix}  
\end{equation}

\begin{remark}\label{rem:lat-int-correspondence}
  An embedding creates the correspondence between a point $\lambda \in \Lambda \subset \R^n$ of an algebraic lattice (Theo.
  \ref{theo:algebraic-lattice}) and an integer in $\Ok$:

  Let $\lambda$ be a point of a lattice $\Lambda$:

\begin{align*} 
     \lambda &= (\lambda_1,\dots,\lambda_{r+2s}) \in \Lambda \\
       &= \left( \sum_{i=1}^n{z_i\sigma_1(\omega_i)} , \dots , \sum_{i=1}^n{z_i\sigma_{r+2s}(\omega_i)} \right) \\
       &= \left( \sigma_1\left(   \sum_{i=1}^n{z_i\omega_i} \right) , \dots , \sigma_{r+2s} \left( \sum_{i=1}^n{z_i\omega_i}  \right) \right) 
\end{align*}
  where $z_i \in \Z$. Since any element $x \in \Ok$ has the form $x =
  \sum_{i=1}^n{\lambda_i\omega_i}$, we can conclude that:

  \begin{equation*}
    \lambda = \left( \sigma_1(x), \dots, \sigma_{r+2s}(x) \right) = \sigma(x)
  \end{equation*}

\end{remark}

\subsection{Twisted Embeddings}
\label{sec:org58db254}

\begin{definition}
Let $K$ be a number field with degree $n$ and $\sigma$ an embedding. We say that a number $\tau \in K$ is
\textbf{totally  positive} if $\forall i \in {1, \dots , n}, \; \sigma_i(\tau) \in \R^*_+$ 
\end{definition}


\begin{definition}[Twisted Embedding]
  \label{def:twisted-emb}
  Given $\tau$ a totally positive number, the \textbf{$\tau$-twisted embedding}, or
  simply twisted embedding, is the monomorphism defined as:
  \begin{equation*}
    \sigma_\tau(x) = \left( \sqrt{\tau_1}\sigma_1(x), \dots, \sqrt{\tau_{r+2s}}\sigma_{r+2s}(x) \right)
  \end{equation*}

  where $\tau_i = \sigma_i(\tau)$.
\end{definition}

\chapter{Objectives}
\label{sec:org084f84e}

\chapter{Methodology}
\label{sec:orgc08c54a}

\bibliographystyle{plain}
\bibliography{ic-tese-v3}
\end{document}