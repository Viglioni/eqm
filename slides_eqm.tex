% Created 2021-03-10 Wed 23:15
% Intended LaTeX compiler: pdflatex
\documentclass[notheorems, bigger]{beamer}
\usepackage[utf8]{inputenc}
\usepackage[T1]{fontenc}
\usepackage{graphicx}
\usepackage{grffile}
\usepackage{longtable}
\usepackage{wrapfig}
\usepackage{rotating}
\usepackage[normalem]{ulem}
\usepackage{amsmath}
\usepackage{textcomp}
\usepackage{amssymb}
\usepackage{capt-of}
\usepackage{hyperref}
%
% American mathematical society libs
%

% imports
\usepackage{amsmath,amssymb,latexsym,amsfonts,amsthm, mathtools}

% definitions
\newtheorem{theorem}{Theorem}[section]
\newtheorem{lemma}{Lemma}[section]
\newtheorem{proposition}{Proposition}[theorem]
\newtheorem{corollary}{Corollary}[theorem]

\theoremstyle{definition}
\newtheorem{definition}{Definition}[section]
\newtheorem{example}{Example}[section]
\newtheorem{remark}{Remark}[section]

%
% Commands and aliases
%

% Complex numbers set
\newcommand{\C}{\mathbb{C}}

% Real numbers set
\newcommand{\R}{\mathbb{R}}
\newcommand{\Rn}{\mathbb{R}^n}


% Rational numbers set
\newcommand{\Q}{\mathbb{Q}}

% Integer numbers set
\newcommand{\Z}{\mathbb{Z}}

% Natural numbers set
\newcommand{\N}{\mathbb{N}}

% Prime numbers set
\renewcommand{\P}{\mathbb{P}}

% Ring of integers set
\renewcommand{\O}{\mathcal{O}}
\newcommand{\Ok}{\mathcal{O}_K}

% Ring of integers set
\newcommand{\Id}{\mathfrak{I}}

% Canonical basis
\newcommand{\Cb}{\mathcal{C}}


% Definition
\newcommand{\defsym}{\vcentcolon =}

% Id Est i.e.
\newcommand{\ie}{\textit{i.e.}}

% Maximum real subfield
\newcommand{\maxrs}{\Q(\zeta_{p} + \zeta_{p} ^{-1})}

% K_\R inner space
\newcommand{\krspace}{(K_\R,\langle{\cdot,\cdot}\rangle_\tau)}
\usetheme{metropolis}
\author{Candidate: Laura Viglioni \\ Supervisor: Prof. Dr. Ricardo Dahab}
\date{March 12, 2021}
\title{A study of some practical impacts of twisted embeddings in lattice-based cryptography}
\hypersetup{
 pdfauthor={Candidate: Laura Viglioni \\ Supervisor: Prof. Dr. Ricardo Dahab},
 pdftitle={A study of some practical impacts of twisted embeddings in lattice-based cryptography},
 pdfkeywords={},
 pdfsubject={},
 pdfcreator={Emacs 27.1 (Org mode 9.4.4)}, 
 pdflang={English}}
\begin{document}

\maketitle

\section{Twisted embeddings}
\label{sec:org41924c3}
\begin{frame}[label={sec:orge3b9d7d}]{Lattices}
\end{frame}
\section{Objectives}
\label{sec:org0d83388}
\begin{frame}[label={sec:orge9256ec}]{Main goal}
\begin{itemize}
\item Validate the idea of using twisted embeddings in cryptography
\item Explore the theoretical and the practical aspects of this proposal
\end{itemize}
\end{frame}
\begin{frame}[label={sec:org969b9b2}]{Practical aspects}
\begin{itemize}
\item Compare implementations and instances of the Twisted Ring-LWE and Ring-LWE
\item Maximum realsubfield versus the cyclotomic power-of-tw
\item Search for proper sizes of keys and messages
\end{itemize}
\end{frame}
\begin{frame}[label={sec:orgf5fcd15}]{Theoretical aspects}
\begin{itemize}
\item Study the polynomial arithmetic of themaximal real subfield
\item Study the relation between the orthonormal basis and the efficient conversion between latticepoints and elements of number field
\item Examine if it is possible toachieve a satisfactory efficiency with non-orthonormal basis
\end{itemize}
\end{frame}
\section{Methodology and timeline}
\label{sec:org69fadbd}
\begin{frame}[label={sec:org37b6460}]{Methodology}
\begin{itemize}
\item \textbf{Literature Review:} review proposals of new cryptosystems, such as \emph{NTTRU}.
\item \textbf{Theoretical experiments:} perform experiments using algebra
  libraries to discover twist factors and to discover orthonormal bases.
\item \textbf{Experimental outcome:} to calculate the expansion factor of the polynomial \(f(x)\) that defines the ring \(\Z[x]/f(x)\). Adapt or develop algorithms for polynomial multiplication.
\item \textbf{Implementation:} implement a Twisted Ring-LWE based cryptosystem.
\item \textbf{Practical experiments:} to estimate the cost in terms of clock cycles, also key and message sizes.
\end{itemize}
\end{frame}
\begin{frame}[label={sec:org4d3a80f}]{Timeline}
\begin{itemize}
\item First and second semesters of 2021
\begin{itemize}
\item Study the Twisted Ring LWE problem and implementation.
\item Perform theoretical experiments with number fields, twist factors and lattices.
\item Calculate the expansion factor and adapt/develop algorithms for polynomial multiplication.
\end{itemize}
\item First and second semesters of 2022
\begin{itemize}
\item Implement a Twisted Ring-LWE based cryptosystem.
\item Compare instances of Ring LWE and Twisted Ring LWE, \ie, analyze the cryptosystem in both terms of clock cycles and key sizes.
\item Defense of dissertation.
\end{itemize}
\end{itemize}
\end{frame}
\section{Thank you!}
\label{sec:org0a33feb}
\end{document}