% Escolha: Portugues ou Ingles ou Espanhol.
% Para a versão final do texto, após a defesa, acrescente Final:

\documentclass[Ingles]{ic-tese-v3}
%\documentclass[Portugues,Final]{ic-tese-v3}

\usepackage[latin1,utf8]{inputenc}

% Para acrescentar comentários ao PDF descomente:
\usepackage
%  [pdfauthor={nome do autor},
%   pdftitle={titulo},
%   pdfkeywords={palavra-chave, palavra-chave},
%   pdfproducer={Latex with hyperref},
%   pdfcreator={pdflatex}]
{hyperref}


\begin{document}


% Escolha entre autor ou autora:
\autora{Laura Viglioni}

% Sempre deve haver um título em português:
\titulo{Título da Dissertação ou Tese em Português}

% Se a língua for o inglês ou o espanhol defina:
\title{The Dissertation or Thesis Title in English }


\orientador{Prof. Dr. Ricardo Dahab}


% Escolha entre coorientador ou coorientadora, se houver.  Inclua os títulos acadêmicos:
%\coorientador{Prof. Dr. Eng. Lic. Nome do Co-Orientador}
%\coorientadora{Profa. Dra. Eng. Lic. Nome da Co-Orientad
% \end{document}
% ora}

% Escolha entre mestrado ou doutorado:
\mestrado
%\doutorado

% Se houve cotutela, defina:
%\cotutela{Universidade Nova de Plutão}

\datadadefesa{30}{02}{2020}

% Para a versão final defina:
%\avaliadorA{Prof. Dr. Primeiro Avaliador}{Instituição do primeiro avaliador}
%\avaliadorB{Profa. Dra. Segunda Avaliadora}{Instituição da segunda avaliadora}
%\avaliadorC{Dr. Terceiro Avaliador}{Instituição do terceiro avaliador}
%\avaliadorD{Prof. Dr. Quarto Avaliador}{Instituição do quarto avaliador}
%\avaliadorE{Prof. Dr. Quinto Avaliador}{Instituição do quinto avaliador}
%\avaliadorF{Prof. Dr. Sexto Avaliador}{Instituição do sexto avaliador}
%\avaliadorG{Prof. Dr. Sétimo Avaliador}{Instituição do sétimo avaliador}
%\avaliadorH{Prof. Dr. Oitavo Avaliador}{Instituição do oitavo avaliador}


% Para incluir a ficha catalográfica em PDF na versão final, descomente e ajuste:
%\fichacatalografica{arquivo.pdf}


% Este comando deve ficar aqui:
\paginasiniciais 


% Se houver dedicatória, descomente:
%\prefacesection{Dedicatória}
%A dedicatória deve ocupar uma única página.


% Se houver epígrafe, descomente e edite:
% \begin{epigrafe}
% {\it
% Vita brevis,\\
% ars longa,\\
% occasio praeceps,\\
% experimentum periculosum,\\
% iudicium difficile.}
%
% \hfill (Hippocrates)
% \end{epigrafe}


% Agradecimentos ou Acknowledgements ou Agradecimientos
% \prefacesection{Agradecimentos}
% Os agradecimentos devem ocupar uma única página.


% Sempre deve haver um resumo em português:
\begin{resumo}
O resumo deve ter no máximo 500 palavras e deve ocupar uma única página.
\end{resumo}


% Sempre deve haver um abstract:
\begin{abstract}
The abstract must have at most 500 words and must fit in a single page.
\end{abstract}


% Se houver um resumo em espanhol, descomente:
%\begin{resumen}
% A mesma regra aplica-se.
%\end{resumen}


% A lista de figuras é opcional:
%\listoffigures

% A lista de tabelas é opcional:
%\listoftables

% A lista de abreviações e siglas é opcional:
% \prefacesection{Lista de Abreviações e Siglas}

% A lista de símbolos é opcional:
% \prefacesection{Lista de Símbolos}

% Quem usa o pacote nomencl pode incluir:
%\renewcommand{\nomname}{Lista de Abreviações e Siglas}
%\printnomenclature[3cm]


% O sumário vem aqui:
\tableofcontents


% E esta linha deve ficar bem aqui:
\fimdaspaginasiniciais


% O corpo da dissertação ou tese começa aqui:
\chapter{Introduction}

\chapter{Mathematical Background}

\section{Groups}
\begin{definition}
  A \textbf{group} is a set $G$ closed under a binary operation $\cdot$ defined on $G$ such
  that:
  \begin{itemize}
  \item \textbf{Associativity: } $\forall a,b,c \in G, \; a\cdot(b\cdot c) = (a\cdot b)\cdot c$
  \item \textbf{Identity element: } $\exists e \in G \; ; \; \forall a \in G, \; a\cdot e = e\cdot a = a$
  \item \textbf{Inverse element: } $\forall a \in G, \; \exists b \in G \; ; \; a\cdot b = b \cdot a = e$
  \end{itemize}
And it is denoted by $\langle G,\cdot\rangle$, or simply $G$ if the operation is implied.
\end{definition}

\begin{definition}
  A group is said to be \textbf{commutative} or \textbf{abelian}
  if $\forall a, b \in G, \; a\cdot b = b\cdot a$
\end{definition}

\noindent
A group is called \textbf{additive} or \textbf{multiplicative} if its
operation is addition or multiplication, respectively.

\begin{definition}
  A subset $H$ of $G$ is a \textbf{subgroup} of $\langle G,\cdot \rangle$ if it is
  closed under $\cdot$ induced by $\langle G,\cdot \rangle$.
\end{definition}

\begin{definition}
  The \textbf{order} of a group $\langle G,\cdot\rangle$ is the cardinality of the set $G$.
\end{definition}

\begin{definition}
  A subgroup $H$ of $G$ can be used to decompose $G$ in uniform sized and
  disjoints subsets called \textbf{cosets}. Given an element $g \in G$:
  \begin{itemize}
  \item A \textbf{left coset} is defined by $gH := \{g\cdot h \; ; \; h \in H\}$
  \item A \textbf{right coset} is defined by $Hg := \{h\cdot g \; ; \; h \in H\}$
  \end{itemize}
\end{definition}

\section{Rings and Fields}

\begin{definition}
  A \textbf{ring} is a set together with two binary operations, we will note by
  $+$ and $*$ and call it addition and multiplication, respectively, such that:
  \begin{itemize}
  \item $\langle R,+\rangle$ is an abelian group.
  \item $*$ is associative
  \item $*$ is distributive over $+$
  \end{itemize}

  And it is denoted by $\langle R,+,*\rangle$, or simply $G$ if the operations are implied.
\end{definition}

\begin{definition}
  A ring is said to be \textbf{commutative} if its $*$ operation is commutative.
\end{definition}

\begin{definition}
  A ring is said to be \textbf{with unity} if $*$ has a identity element. We
  shall note it by $1$ and it is called \textbf{unity}.
  
\end{definition}

\begin{definition}
  A \textbf{division ring} is a ring R where $\forall r \in R, \; \exists s \in R \; ; \; r*s = 1$.
\end{definition}

\begin{definition}
  A \textbf{field} is a commutative division ring.
\end{definition}

\section{Lattices}
\label{sec:lattices}

\begin{definition}
  A Lattice $\Lambda \subset \mathbb{R}^n$ is a subgroup of the additive group $\mathbb{R}^n$
\end{definition}

In other words, given $m$ linear independent vectors in $\mathbb{R}^n$, the set
$\{v_1, v_2, ..., v_m\}$ is called a \textbf{basis} for $\Lambda$ and the Lattice may defined
by:

\begin{definition}
  \begin{equation}
    \Lambda := \left\{x = \sum_{i=1}^m{\lambda_iv_i} \in \mathbb{R}^n \; | \; \lambda_i \in \mathbb{Z}\right\}
  \end{equation}
\end{definition}

\section{Number Fields}
\label{sec:number-fields}

\begin{definition}
  Let $K$ and $L$ be two fields, $L$ is said to be a \textbf{field extension} of
  $K$ if $L \subseteq K$ and we denote it by $L/K$
\end{definition}

Note that in a field extension $L/K$, $L$ has a structure of a vector space over
$K$, where vector addition is in $L$ and scalar multiplication $a \in K, \; v \in L
\; \implies av \in L$. The dimension of $L$ as a vector space is called
\textbf{degree} and it is denoted by $[L:K]$.

\begin{definition}
  A field extension is called \textbf{number field} when it is over $\mathbb{Q}$.
\end{definition}

\begin{definition}
  Let $\alpha \in L$ where $L/K$ is a field extension. We say that $\alpha$ is
  \textbf{algebraic over $K$} if $\exists p \in K[X] \;;\; p(\alpha) = 0$. $p$ is said to be
  \textbf{the minimal polynomial of $\alpha$ over $K$} denoted by $p_\alpha$. If $\alpha \in L =
  \mathbb{Q}[\theta]$, we simply call $\alpha$ an \textbf{algebraic number}.
\end{definition}

\begin{example}
  It is known that $\mathbb{Q}$ is a field. If we add $\sqrt{2}$ to the set, we
  can build a new field adding also all the powers and multiples of
  $\mathbb{Q}$. This new field is denoted by $\mathbb{Q}[\sqrt{2}]$, note that
  $\sqrt{2}$ is algebraic and its minimal polynomial $p_{\sqrt{2}} = x^2-2$. All
  elements of $\mathbb{Q}[\sqrt{2}]$ are in the form $\{a+b\sqrt{2} \;|\; a,b \in
  \mathbb{Q}\}$ and one of its basis is $\{1, \sqrt{2}\}$, so it has degree is
  $2$.
\end{example}

\begin{example}
  If we add $\sqrt[3]{2}$ to $\mathbb{Q}$ instead, its elements would have the
  form $\{a + b\sqrt[3]{2} + c\sqrt[3]{4} \;|\; a,b,c \in \mathbb{Q}\}$, so one of
  its basis is $\{1 ,\sqrt[3]{2} ,\sqrt[3]{4}\}$, $p_\alpha = x^3 - 2$ and its degree
  is $3$.
\end{example}

\begin{theorem}
  [add font 45 p.40] If $K$ is a number field, then $K = \mathbb{Q}[\theta]$ for some
  algebraic number $\theta \in K$, called primitive element.
\end{theorem}

Then we conclude that $\{1, \theta, \theta^2, ... , \theta^{n-1}\}$ is a basis for the vector
space $K = \mathbb{Q}[\theta]$ over $\mathbb{Q}$.

\section{Twisted Embedding}
\label{sec:twisted-embeddding}



\chapter{Hypothesis}


\chapter{Results}


\chapter{Conclusions}



% As referências:
\bibliographystyle{plain}
\bibliography{ic-tese-v3}


% Os anexos, se houver, vêm depois das referências:
%\appendix

\end{document}
